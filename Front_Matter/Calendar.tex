\thispagestyle{plain}
\begin{fullwidth}
\begin{center}
{\noindent\LARGE\textsc{Cronograma}} \\
\end{center}
\end{fullwidth}

\vspace{1cm}
\begin{fullwidth}
\it
As aulas seguirão o planejamento abaixo. No calendário ao lado, estão circuladas as datas das provas.
\end{fullwidth}

%%%
% Engenharia Mecânica
%%%

\begin{marginfigure}[4cm]
    \centering
    Março\\
    \begin{tikzpicture}
        \calendar (mycal)
        [
            dates=2022-03-01 to 2022-03-last,
            week list,
            day headings=gray,
            day letter headings
        ]
        if (Saturday,Sunday)
            [gray]
        if (at most=2022-03-02)
            [gray]
        ;
    \end{tikzpicture}
\end{marginfigure} %
%
\begin{marginfigure}
    \centering
    Abril\\
    \begin{tikzpicture}
        \calendar (mycal)
        [
            dates=2022-04-01 to 2022-04-last,
            week list,
            day headings=gray,
            day letter headings
        ]
        if (Saturday,Sunday)
            [gray]
        if (equals=2022-04-15, equals=2022-04-21, equals=2022-04-22)
            [gray]
        ;
    \end{tikzpicture}
\end{marginfigure} %
%
\begin{marginfigure}
    \centering
    Maio\\
    \begin{tikzpicture}
        \calendar (mycal)
        [
            dates=2022-05-01 to 2022-05-last,
            week list,
            day headings=gray,
            day letter headings
        ]
        if (Saturday,Sunday)
            [gray]
        if (equals= 2022-06-16, equals=2022-06-17, equals=2022-06-29)
            [gray]
        ;
    \end{tikzpicture}
\end{marginfigure} %
%
\begin{marginfigure}
    \centering
    Junho\\
    \begin{tikzpicture}
        \calendar (mycal)
        [
            dates=2022-06-01 to 2022-06-last,
            week list,
            day headings=gray,
            day letter headings
        ]
        if (Saturday,Sunday)
            [gray]
        ;
        \draw (mycal-2022-06-23) circle (6pt);
    \end{tikzpicture}
\end{marginfigure} %
%
\begin{marginfigure}
    \centering
    Julho\\
    \begin{tikzpicture}
        \calendar (mycal)
        [
            dates=2022-07-01 to 2022-07-last,
            week list,
            day headings=gray,
            day letter headings
        ]
        if (Saturday,Sunday)
            [gray]
        if (at most=2022-07-06)
            {}
        else
            [gray]
        ;
    \end{tikzpicture}
\end{marginfigure}
\vspace{1cm}
\begin{center}
\Large\textsc{Engenharia Mecânica}
\end{center}

As aulas seguirão o planejamento abaixo.
\begin{center}
\begin{longtable}{ccp{70mm}}
\toprule
Aula & Data & Conteúdo \\
\midrule
\endhead
\bottomrule
\endfoot
 1 & 08/03 & Revisão de conceitos: Elaboração de gráficos de dados experimentais. \\
 2 & 15/03 & Unidades: Sistemas de unidades, conversão de unidades. \\
 3 & 22/03 & Turma A: Exp. 1, Medidas. \\
 4 & 29/03 & Turma B: Exp. 1, Medidas. \\ 
 5 & 05/04 & Turma A: Exp. 2, MRU e MRUV. \\
 6 & 12/04 & Turma B: Exp. 2, MRU e MRUV. \\
 7 & 19/04 & Turma A: Exp. 3, Lei de Hooke. \\
 8 & 26/04 & Turma B: Exp. 3, Lei de Hooke. \\
 9 & 03/05 & Turma A: Exp. 4, Leis de Newton. \\
10 & 10/05 & Turma B: Exp. 4, Leis de Newton. \\
11 & 17/05 & Turma A: Exp. 5, Atrito. \\
12 & 24/05 & Turma B: Exp. 5, Atrito. \\
13 & 31/05 & Turma A: Exp. 6, Arrasto. \\
14 & 07/06 & Turma B: Exp. 6, Arrasto. \\
15 & 14/06 & Turma A: Exp. 7, Trabalho e Energia. \\
16 & 21/06 & Turma B: Exp. 7, Trabalho e Energia. \\
17 & 28/06 & Turmas A e B: Prova. \\
18 & 05/07 & Entrega das notas finais de laboratório. \\
\end{longtable}
\end{center}

\clearpage

\begin{marginfigure}[2cm]
    \centering
    Março\\
    \begin{tikzpicture}
        \calendar (mycal)
        [
            dates=2022-03-01 to 2022-03-last,
            week list,
            day headings=gray,
            day letter headings
        ]
        if (Saturday,Sunday)
            [gray]
        if (at most=2022-03-02)
            [gray]
        ;
    \end{tikzpicture}
\end{marginfigure} %
%
\begin{marginfigure}[2.5mm]
    \centering
    Abril\\
    \begin{tikzpicture}
        \calendar (mycal)
        [
            dates=2022-04-01 to 2022-04-last,
            week list,
            day headings=gray,
            day letter headings
        ]
        if (Saturday,Sunday)
            [gray]
        if (equals=2022-04-15, equals=2022-04-21, equals=2022-04-22)
            [gray]
        ;
    \end{tikzpicture}
\end{marginfigure} %
%
\begin{marginfigure}
    \centering
    Maio\\
    \begin{tikzpicture}
        \calendar (mycal)
        [
            dates=2022-05-01 to 2022-05-last,
            week list,
            day headings=gray,
            day letter headings
        ]
        if (Saturday,Sunday)
            [gray]
        if (equals= 2022-06-16, equals=2022-06-17, equals=2022-06-29)
            [gray]
        ;
    \end{tikzpicture}
\end{marginfigure} %
%
\begin{marginfigure}
    \centering
    Junho\\
    \begin{tikzpicture}
        \calendar (mycal)
        [
            dates=2022-06-01 to 2022-06-last,
            week list,
            day headings=gray,
            day letter headings
        ]
        if (Saturday,Sunday)
            [gray]
        ;
        \draw (mycal-2022-06-23) circle (6pt);
    \end{tikzpicture}
\end{marginfigure} %
%
\begin{marginfigure}
    \centering
    Julho\\
    \begin{tikzpicture}
        \calendar (mycal)
        [
            dates=2022-07-01 to 2022-07-last,
            week list,
            day headings=gray,
            day letter headings
        ]
        if (Saturday,Sunday)
            [gray]
        if (at most=2022-07-06)
            {}
        else
            [gray]
        ;
    \end{tikzpicture}
\end{marginfigure}
\vspace{1cm}
\begin{center}
\Large\textsc{Engenharia da Computação}
\end{center}

As aulas seguirão o planejamento abaixo.
\begin{center}
\begin{longtable}{ccp{70mm}}
\toprule
Aula & Data & Conteúdo \\
\midrule
\endhead
\bottomrule
\endfoot
 1 & 03/03 & Pré-apresentação. \\
 2 & 10/03 & Revisão de conceitos: Elaboração de gráficos de dados experimentais. \\
 3 & 17/03 & Turma A: Exp. 1, Medidas. \\
 4 & 24/03 & Turma B: Exp. 1, Medidas. \\ 
 5 & 31/03 & Turma A: Exp. 2, MRU e MRUV. \\
 6 & 07/04 & Turma B: Exp. 2, MRU e MRUV. \\
 7 & 14/04 & Turma A: Exp. 3, Lei de Hooke. \\
-- & 21/04 & \emph{Feriado}. \\
 8 & 28/04 & Turma B: Exp. 3, Lei de Hooke. \\
 9 & 05/05 & Turma A: Exp. 4, Leis de Newton. \\
10 & 12/05 & Turma B: Exp. 4, Leis de Newton. \\
11 & 19/05 & Turma A: Exp. 5, Atrito. \\
12 & 26/05 & Turma B: Exp. 5, Atrito. \\
13 & 02/06 & Turma A: Exp. 6, Arrasto. \\
14 & 09/06 & Turma B: Exp. 6, Arrasto. \\
-- & 16/06 & \emph{Feriado}.\\
15 & 23/06 & Turmas A e B: Prova. \\
16 & 30/06 & Entrega das notas finais de laboratório.
\end{longtable}
\end{center}

\cleardoublepage
